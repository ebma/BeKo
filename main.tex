\documentclass[10pt,leqno ]{article}

\usepackage{amsfonts}
\usepackage{amssymb}
\usepackage{array}
\usepackage{amsmath}
\usepackage{times}
\usepackage{mathtools}
\usepackage{amsthm}
\usepackage{hyperref}
\usepackage[margin=1.5in]{geometry}
\usepackage{setspace}
\usepackage[german]{babel} %Deutsche Sprachunterstützung
\usepackage[babel,german=quotes]{csquotes} % Für gute Anführungszeichen

\newcommand{\estimates}{\overset{\scriptscriptstyle\wedge}{=}}

\begin{document}

\section{Rekursion}
\subsection{Binomialkoeffizient}
Zuerst definieren wir eine $\mu$-rekursive Fakultätsfunktion $fak: \mathbb{N} \rightarrow \mathbb{N}$ mit:
\begin{align*}
    fak(0) =& c^0_1 \\
    fak(n+1) =& mult \circ (succ(\Pi^2_1), \Pi^2_2)(n, fak(n))
\end{align*}

Da die zur Definition verwendete $mult$-Funktion als $\mu$-rekursiv angenommen werden kann, und alle weiteren Bestandteile Inhalt der Klasse der primitiv rekursiven Funktionen sind, ist die Funktion $fak$ somit auch $\mu$-rekursiv.

Mithilfe dieser Fakultätsfunktion können wir nun die $\mu$-rekursive Funktion $g: \mathbb{N} \times \mathbb{N} \rightarrow \mathbb{N}$ mit: 
\begin{align*}
    g(n,k) = div \circ (fak \circ \Pi^2_1, mult \circ (fak \circ \Pi^2_2, fak \circ sub(\Pi^2_1, \Pi^2_2))) (n,k) \\
\end{align*}
definieren.

Da wir vorher gezeigt haben, dass die Funktion $fak$ $\mu$-rekursiv ist, und jede Funktion, die durch Einsetzung/Komposition von $\mu$-rekursiven Funktionen entsteht, auch $\mu$-rekursiv ist, ist die Funktion $g$ selbst $\mu$-rekursiv.


\subsection{}
Die geschlossene Formel für $f(1,x,y)$ für $y \le x$ lautet: 
\[ 2^x * \frac{x!}{y! * (x - y)!} \]

\subsubsection*{Beweis:}
\begin{flushleft}
    Wir führen Induktion über x:
\end{flushleft}

Induktionsanfang:
Für $x = 0$ gilt $y = 0$, da $y \le x$ und 
\begin{equation*}
f(1,0,0) = 1 = 2^0 * \frac{0!}{0! * (0-0)!}
\end{equation*}

Induktionsvoraussetzung:
\begin{equation*}
f(1,x,y) = 2^x * \frac{x!}{y! * (x - y)!}
\end{equation*}

Induktionsbehauptung:
\begin{equation*}
f(1,x+1,y) = 2^{x+1} * \frac{(x+1)!}{y! * ((x+1) - y)!}
\end{equation*}

Induktionsschluss:
\begin{align*}
f(1,x+1,y) &= f(0, f(1,x,y-1), f(1,x,y)) \\
           &= 2 * f(1,x,y-1) + 2 * f(1,x,y) \\
           &= 2 * 2^x * \frac{x!}{(y-1)! * (x - (y-1))!} + 2 * 2^x * \frac{x!}{y! * (x - y)!} \\ % TODO add I.V. above '='
           &= 2^{x+1} * (\frac{x!}{(y-1)! * (x - (y-1))!} + \frac{x!}{y! * (x - y)!}) \\
           &= 2^{x+1} * ( \binom{x}{y-1} + \binom{x}{y}) \\
           &= 2^{x+1} * ( \binom{x+1}{y}) \\
           &= 2^{x+1} * \frac{(x+1)!}{y! * ((x+1) - y)!}
\end{align*}

\begin{flushleft}
    Wir führen Induktion über y:
\end{flushleft}

Induktionsanfang:
Für y = 0 gilt:
\begin{align*}
f(1,x,0) &= f(0, 0, f(1, x-1, 0)) \\
         &= 2*f(1,x-1,0) \\
         &= ... \\
         &= 2^x * f(1,0,0) \\
         &= 2^x * 1 \\
         &= 2^x * \frac{x!}{x!} \\
         &= 2^x * \frac{x!}{0! * (x-0)!} \\
\end{align*}

Induktionsvoraussetzung:
\begin{equation*}
f(1,x,y) = 2^x * \frac{x!}{y! * (x - y)!}
\end{equation*}

Induktionsbehauptung:
\begin{equation*}
f(1,x,y+1) = 2^{x} * \frac{(x)!}{(y+1)! * ((x) - (y+1)!}
\end{equation*}

Induktionsschluss:
\begin{align*}
f(1,x,y+1) &= f(0, f(1,x-1,y), f(1,x-1,y+1)) \\
           &= 2 * f(1,x-1,y) + 2 * f(1,x-1,y+1) \\
           &= 2 * 2^{x-1} * \frac{(x-1)!}{y! * ((x-1) - y)!} + 2 * 2^{x-1} * \frac{(x-1)!}{(y+1)! * ((x-1) - (y+1))!} \\ % TODO add I.V. above '='
           &= 2^x * (\frac{(x-1)!}{y! * ((x-1) - y)!} + \frac{(x-1)!}{(y+1)! * ((x-1) - (y+1))!}) \\
           &= 2^x * ( \binom{x-1}{y} + \binom{x-1}{y+1}) \\
           &= 2^x * ( \binom{x}{y+1}) \\
           &= 2^x * \frac{(x)!}{(y+1)! * ((x) - (y+1)!}
\end{align*}

\vspace{1cm}
\begin{flushleft}
    Aus den beiden durchgeführten Induktionsbeweisen folgt, dass die Formel 
    \[ 2^x * \frac{x!}{y! * (x - y)!} \]
    für $f(1,x,y)$ für $y \le x$ gilt.
\end{flushleft}

\pagebreak

\section{Simulation}

Simulation von WHILE durch LOOP*

\paragraph{Hinweis}
Zeigen Sie zunächst, dass LOOP-Programme mit dem zusätzlichen Konstrukt
Turing-mächtig sind.
    
LOOP-Programme sind mithilfe des zusätzlichen Konstruktes \textit{DO P WHILE  $x_i \neq 0$ END} Turing-mächtig, da jedes WHILE-Programm \textit{WHILE $x_i \neq 0$ DO P END} durch das folgende LOOP-Programm (mit Konstrukt) simuliert werden kann:

\begin{flushleft}
IF $x_i \neq 0$ THEN \\ 
\quad    DO P WHILE $x_i \neq 0$ END; \\ 
END
\end{flushleft}

Im Folgenden wird gezeigt, dass LOOP*-Programme Turing-mächtig sind, indem bewiesen wird, wie jedes WHILE-Programm durch ein LOOP*-Programm simuliert werden kann.


\paragraph{Definition Hilfskonstrukt}
Zunächst soll ein Hilfskonstrukt eingeführt werden, welches durch ein LOOP-Programm simuliert werden kann.
\begin{flushleft}
    IF $x_i \neq 0$ THEN P END $\estimates x_i := 0$; LOOP $x_i$ DO $x_j := 1$ END; LOOP $x_j$ DO P END;
\end{flushleft}

\subsubsection*{Beweis}
Ein jedes WHILE-Programm \textit{WHILE $x_i \neq 0$ DO P END} kann durch das folgende LOOP*-Programm simuliert werden:

\begin{flushleft}
    IF $x_i \neq 0$ THEN \\ 
    \quad $x_{N_P} := x_0;$ \\ 
    \quad MAKE P  BREAK \\
    \qquad IF $x_i \neq 0$ THEN \\ 
    \quad \qquad $x_0 := 0;$ \\
    \qquad END; \\
    \quad END; \\ 
    \quad $x_0 := x_{N_P}$ \\
    END;
\end{flushleft}

Durch die IF-Anweisung in der ersten Zeile wird sichergestellt, dass der \textit{MAKE}-Block nur dann ausgeführt wird, wenn $x_i \neq 0$ ist. Das ist wichtig, da bei WHILE-Programmen die Bedingung auch zuerst überprüft wird, bevor der Block das erste Mal ausgeführt wird.

Die Zuweisung des Wertes von $x_0$ an $x_{N_P}$ vor dem \textit{MAKE}-Block und wieder zurück nach Ausführung des \textit{MAKE}-Blocks stellt sicher, dass der zuvor in $x_0$ gespeicherte Wert nicht verloren geht. Da zum Speichern des Wertes von $x_0$ die Variable $x_{N_P}$ verwendet wird, ist sichergestellt, dass keine in P benutzte Variable überschrieben wird.

\pagebreak

\section{Unentscheidbarkeit}

Sei 
\begin{align*}
    P := \{ m_1 \# m_2 \# w \mid \exists M_1 : m_1 = \langle M_1 \rangle , \exists M_2 : m_2 = \langle M_2 \rangle \text{ und } \\ M_1 \text{ und } M_2 \text{ verhalten sich unterschiedlich auf Eingabewort w} \}
\end{align*}
die Menge aller Wörter, welche die Frage \enquote{Verhalten sich $M$ und $M'$ bei Eingabewort $w$ unterschiedlich?} mit \enquote{Ja} beantworten.

\begin{flushleft}
    Wir zeigen, dass P nicht entscheidbar ist, in dem wir zeigen, dass 
    \[ K = \{ w \in \{0,1 \}^* \mid M_w \text{ hält auf Eingabe } w \} \]
    auf P reduzierbar ist.
\end{flushleft}

\begin{flushleft}
    Zu zeigen ist $K \leq P$.
\end{flushleft}

\begin{flushleft}
    Sei $f: \{ 0, 1 \}^* \rightarrow \{ 0,1,\# \}^*$ mit: \( f(w) = w\#w\#w \)
\end{flushleft}

\begin{flushleft}
    $f$ ist total, da nach Definition $M_w$ für jedes $w$ eine Turing-Maschine $M_w$ existiert. $f$ ist berechenbar, da nach der Vorlesung $M_w$ berechenbar ist, eine universelle Turing-Maschine und eine Turing-Maschine die das Band löscht existieren und die Kodierung einer Turing-Maschine berechenbar ist.
    Wenn des weiteren
    \[... \Leftrightarrow ... \]
    gilt, ist die Reduktion gezeigt.
\end{flushleft}


\begin{flushleft}
    Wenn also K entscheidbar wäre, dann musste auch P entscheidbar sein. Da dies nicht der Fall ist, ist auch P nicht entscheidbar. 
\end{flushleft}

\pagebreak

\section{Ackermann-Funktion}


\end{document}